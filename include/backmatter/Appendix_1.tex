% CREATED BY DAVID FRISK, 2018
\chapter{Appendix 1}

\section{Algorithm for modifying the line thickness}
\label{app:1}
Let $I(x,y)$ denote a two dimensional image and a structuring element of radius $r$. Kozielski et. al. \cite{kozielski2012moment} described the operation of making the lines in the image thicker as a morphological dilation. A dilated image $I'(x,y)$ fulfills is calculated using eq. \eqref{eq:dilate}

\begin{equation}
    \label{eq:dilate}
    I'(x,y) = \max_{r_x,r_y : d(r_x,r_y) < r} I(x+r_x,y+r_y) \text{ for } r \geq 0
\end{equation}

If $r \leq 0$ then the operation becomes that of a thinning operation that is a variation of morphological erosion. Described in eq \eqref{eq:erode}

\begin{equation}
    \label{eq:erode}
    I'(x,y) = \min_{r_x,r_y : d(r_x,r_y) < r} I(x+r_x,y+r_y) \text{ for } r \geq 0
\end{equation}

Given these operations, the line thickness $\tau$ of for example an image $I(x,y)$ of a handwritten digit can be calculated using \eqref{eq:lt} where $G(I)$ denotes the image gradient of $I(x,y)$ which is defined by \cite{kozielski2012moment} as the difference $I_{thick} - I_{thin}$ between a thinned image $I_{thin}$ and a thicked image $I_{thick}$ by a structuring element of radius 1.
\begin{equation}
    \label{eq:lt}
    \tau = 2\frac{\sum_x\sum_yI(x,y)}{\sum_x\sum_yG(I)(x,y)}
\end{equation}

Suppose that an image $I(x,y)$ is supposed to be adjusted to a target line thickness $T$, then the algorithm can be described using the following steps.

\begin{enumerate}
    \item calculate the radius of a structuring element as $r = T - \tau$ (using the current line thickness).
    \item Apply dilation or erosion on the image using either eq. \eqref{eq:dilate} or \eqref{eq:erode}.
    \item Calculate the new line thickness $\tau_{new}$ using \eqref{eq:lt}.
\end{enumerate}

These steps are repeated until $|\tau_{new} - T| \leq \epsilon$ where epsilon stands for an error division. Throughout our experiments, $\epsilon$ is set to 0.9.